%%
%% VERSION HISTORY
%%    22 May 2006 - John Papandriopoulos - Original version
%%    12 Jul 2007 - John Papandriopoulos - Converted into template
%%    30 Sep 2015 - Maoyuan Liu - Draft version
%%

% -----
% Basic
% -----

% AMS packages
%\usepackage{amsfonts}
\usepackage{amssymb}
\usepackage{amsmath}
\usepackage{amsthm}
\usepackage[mathscr]{eucal}

\usepackage{color}
\usepackage{xcolor}

% Graphics
\usepackage{graphicx}
\usepackage{float}
%\usepackage{caption}

% For subfigures
%\usepackage{subfig}
%\usepackage{subfigure}


% --------------------
% Lists and references
% --------------------
\usepackage{url}
\usepackage[linktoc=all]{hyperref}
\hypersetup{
    colorlinks,
    citecolor=black,
    filecolor=black,
    linkcolor=black,
    urlcolor=black
}

% bibliography at the end of every chapter
\usepackage[numbers,sort&compress,sectionbib,square]{natbib}
\usepackage{chapterbib}
% call \mybib{file} at the end of a chapter to print bibliography
\newcommand{\mybib}[1]{
    \clearpage
    \bibliographystyle{plainnat}
    \bibliography{#1}
}

% List of index
\usepackage{makeidx}
\makeindex

% Glossary --> defitions of symbols and acronyms
\usepackage[toc,nonumberlist,nopostdot]{glossaries}
\makeglossaries
\setglossarystyle{treegroup}

% ------
% Layout
% ------

% Suppress "This page intentionally left blank."
\newcommand{\markblankpages}{}

% set page margins
%   these margins are set by the A4 paper ratios, 1:sqrt(2)
%   paperheight/paperwidth = paperwidth/textwidth = paperheight/textheight = sqrt(2)
%   outer/inner = bottom/top = sqrt(2)
\usepackage[outer=36.02988mm,inner=25.47697mm,top=36.0317mm,bottom=50.9565mm]{geometry}
\newcommand{\symmetricmargin}{
    \newgeometry{outer=30.7534mm,inner=30.7534mm,top=36.0317mm,bottom=50.9565mm}
}
\newcommand{\bookmargin}{
    \newgeometry{outer=36.02988mm,inner=25.47697mm,top=36.0317mm,bottom=50.9565mm}
}

% we've already gone this far, why not go the whole mile
% set the text spacing 
\setstretch{1.414}

% packages to show page margins
%\usepackage{showframe}
%\usepackage{layouts}
\newcommand{\printlayouts}{
  % printing numeric page margins
  \printinunitsof{mm}{\pagevalues}
  \verb|\marginparwidth|: \printinunitsof{mm}\prntlen{\marginparwidth}
  \clearpage
}

% For testing the formatting
\usepackage{lipsum}


% ---------
% Eye candy
% ---------
\RequirePackage[labelfont={bf,singlespacing},
                textfont={singlespacing},
                justification={justified,RaggedRight},
                singlelinecheck=false,
                margin=0pt,
                figurewithin=chapter,
                tablewithin=chapter]{caption}

% fancy chapter titles and quotes at beginning of title
\usepackage[helvetica]{quotchap} 

% Draft watermark
\usepackage[contents={}]{background}
\usepackage[yyyymmdd,hhmmss]{datetime}
\newcommand\DraftText{Draft compiled on \today\ at \currenttime}
\newcommand{\draft}{
  \newcommand{\archivalpapernote}{}
  \backgroundsetup{
    color=lightgray,
    position=current page.center,
    angle=90,
    vshift=0.45\paperwidth,
    opacity=1,
    scale=1,
    contents={\LARGE\ttfamily\DraftText}
  }
}

% Drop caps
%   - using IEEEtrantools
%   - using lettrine (more customizable)
%
%%\usepackage{IEEEtrantools}
%%\newcommand{\PARstart}[2]{\IEEEPARstart{#1}{#2}}
%%
%%\usepackage{lettrine}
%%\renewcommand{\LettrineFontHook}{\fontfamily{ppl}}
%%\setcounter{DefaultLines}{3}
%%\setlength{\DefaultNindent}{0em}

% -----------
% Recycle bin
% -----------
% Some of these may be useful at some point... not using these now though
% 
% Allow equations to break over pages...
%\interdisplaylinepenalty=2500
% Command to stop equation breaks
% Note: enclose this in braces when used...
%\newcommand{\donotsplitoverpages}{\interdisplaylinepenalty=10000}

% For cases
%\usepackage{sublabel}

% For theroem numbers having the chapter included
%\usepackage{style/chngcntr}

% For cool theorem styles
%\usepackage[amsthm]{ntheorem}
%\theorembodyfont{\normalfont}

% Used in the continued list environment below
%\newcounter{continuedlist}
%
%% Continued list environment
%\newenvironment{continuedlist}{%
%  \begin{enumerate}%
%    % Space out each item
%    \setlength{\itemsep}{1.25em}%
%    % Start the enumeration from the previous value
%    \setcounter{enumi}{\value{continuedlist}} 
%}{%
%    % Save the counter to continue it later
%    \setcounter{continuedlist}{\value{enumi}}%
%  \end{enumerate}%
%  %\vspace{1.25em}%
%  \vspace{1em}%
%}

% Spaced out list environment
%\newenvironment{spacedoutlist}{%
%  \begin{itemize}%
%    % Space out each item
%    \setlength{\itemsep}{1.25em}%
%}{%
%  \end{itemize}%
%}

